% Revisão OK 25/09
\chapter{Sistemas Similares}
Os EFBs possuem funções variadas como cálculo de combustível, gasto de combustível
em voo, velocidades de decolagem, etc. Para aeronaves mais novas, como o Airbus 
A320neo, é difícil implementar o cálculo de performance e combustível, pois não 
é disponibilizado ao público como este cálculo é feito. Ferramentas encontradas 
na Internet \cite{a320-perf} normalmente fazem engenharia reversa, e portanto, 
podem apresentar resultados diferentes de um cálculo oficial feito no aplicativo
de tablet.

Nos simuladores de voo para computador pessoal, algumas aeronaves simulam partes
deste quipamento como o Airbus A320neo desenvolvido pela \textit{FlyByWire Simulations}. 
Apesar de ser uma aeronave \textit{freeware}, ela é bem sofisticada chegando ao 
nível de realismo da \textit{Fenix Simulations} ou da \textit{ToLiss Simulations}, 
duas produtoras com modelos pagos do A320.

\begin{figure}[ht]
    \begin{center}
    \includegraphics[width=400pt]{img/efb-a320.png}
    \caption{Exemplo de um EFB no Flight Simulator 2020 na aeronave A320neo}
    \label{fig:efb-a320}
    \end{center}
\end{figure}

Uma das informações importantes para a realização de um voo é o METAR. Trata-se 
de uma string codificada com as condições meteorológicas atuais de um aeródromo. 
Todos os pilotos aprendem a ler um METAR.

Contudo, o METAR do aeródromo não se encontra disponível no EFB. É possível usar
 o computador de bordo da aeronave (FMC) e conseguir esta informação. Também é 
 possível sintonizar na frequência do ATIS, mas isto só funcionará se o avião
 estiver perto do aeródromo.

O que muitos jogadores de simuladores de voo fazem é acessar o AISWEB 
(\url{https://aisweb.decea.mil.br/}), sistema oficial brasileiro de informações
aeronáuticas. 

\begin{figure}[ht]
    \begin{center}
    \includegraphics[width=400pt]{img/aisweb.png}
    \caption{AISWEB com informações de pista, frequências de comunicação e navegação para o Santos Dumont}
    \label{fig:aisweb}
    \end{center}
\end{figure}

É um site extremamente completo e usado em operações reais, mas para o jogador 
iniciante seria de valia uma interface mais simples. O AISWEB exibe o METAR no 
aeroporto, mas não explica para o que cada campo serve.

\begin{figure}[ht]
    \begin{center}
    \includegraphics[width=200pt]{img/metar-aisweb.png}
    \caption{METAR do Santos Dumont no AISWEB}
    \label{fig:aisweb}
    \end{center}
\end{figure}

O site METAR-TAF (\url{https://metar-taf.com/}) é um decoder bem conhecido, possui
uma interface gráfica bem construída e muito fácil de entender, mas não possui a
lista de frequência dos aeroportos e de radionavegação. Como o nome deste sugere,
também é disponibilizado o TAF que é parecido com o METAR, mas o TAF é uma 
previsão das condições.

\begin{figure}[ht]
    \begin{center}
    \includegraphics[width=400pt]{img/ui-metar-taf.png}
    \caption{Interface gráfica do METAR-TAF}
    \label{fig:metar-taf}
    \end{center}
\end{figure}
