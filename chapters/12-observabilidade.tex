% Revisão OK 30/09
\chapter{Observabilidade}

Com o aumento da complexidade do projeto, é importante saber quando algum usuário 
teve problemas ao tentar acessar alguma rota que não funcionou corretamente.
Também é interessante ver quais URLs robôs estão tentando acessar para fechar uma
possível falha de segurança.

O NGINX possui um arquivo de logs localizado em \verb|/var/log/nginx/access.log|. 
No entanto, com o aumento do número de acessos, esse arquivo se torna cada vez 
mais difícil de interpretar.

A solução é utilizar uma ferramenta que processe esse arquivo de log e exiba os 
dados em uma interface gráfica. Para isso, usou-se o GoAccess \cite{goaccess}. Ele 
é um programa para Linux que, ao receber o arquivo de log, gera um arquivo HTML 
como saída.

\lstinputlisting[label=cod
, title={GoAccess}, caption={Comando para executar o GoAccess}, language=bash]{code/goaccess.sh}


\begin{figure}[ht]
    \begin{center}
    \includegraphics[width=400pt]{img/dashboard.png}
    \caption{Dashboard do GoAccess}
    \label{fig:cloudflare-stat.png}
    \end{center}
\end{figure}

Observe que o arquivo de log que estou acessando é \verb|/var/log/nginx/nginx-access.log|. 
Isso foi necessário porque a imagem \textit{Docker} do NGINX que uso, por padrão, cria um 
link simbólico de \verb|/var/log/nginx/access.log| para \verb|/dev/stdout|, o que 
impossibilitava o acesso ao arquivo. Para contornar esse problema, precisou-se 
configurar o arquivo de configuração do NGINX para que os logs de acesso fossem 
gravados também em outro local.

Também é importante destacar que o GoAccess não serve o arquivo HTML gerado 
automaticamente, apenas o cria. A solução foi utilizar o próprio NGINX para 
servir esse arquivo na rota \texttt{a4barros.com/private/report}. Essa URL está 
protegida por autenticação HTTP básica. O arquivo HTML gerado é estático e não é 
atualizado automaticamente. Existe, contudo, a opção de usar o parâmetro 
\texttt{--real-time-html}, que abre um websocket na porta 7890 e adiciona um código 
JavaScript à página HTML para atualizações em tempo real. No entanto, como não 
deseja-se manter essa porta aberta, optou-se por criar um script shell que recria essa
página a cada 3 minutos.

\lstinputlisting[label=cod
, title={GoAccess}, caption={Dockerfile do container do GoAccess}, language=bash]{code/goaccess.Dockerfile}

Em média, o GoAccess consegue processar 101.245,3 linhas de log por segundo 
\cite{goaccess-speed}, então executá-lo a cada três minutos para a quantidade de 
acessos médios (aproximadamente 4640 requisições de 48 IPs únicos por dia) não 
é problema.