\chapter{A Proposta}
É implementado um sistema de auxílio da navegação aérea para intusiastas da
simulação. Pelo fato de aviação necessitar ter um ambiente seguro, considerando
que meu projeto é apenas um protótipó, prefiri restrigir o caso de uso apenas
para jogadores de simuladores de voo que desejam que a simulação seja parecida
com o real. Nas páginas do sistema conterá um aviso que o sistema \textbf{não
deve ser usado para um voo real}.

Dito isto, o sistema é possui backend escrito na linguagem Python fazendo uso
da biblioteca Flask. A renderização de página é server-side, usando os templates
do Flask junto com a biblioteca Jinja2.

O usuário tem acesso a informações de frequência da torre, solo, tráfego, rampa
e operações como também das frequências e dados para VOR, um sistema de radionavegação
e ILS, sistema de pouso por instrumentos. Estas infomações ficaram guardadas
em um banco de dados relacional e neste trabalho será mostrado a Organização
e relação entre as tabelas.

Através de uma API do serviço americano National Weather Service é coletada 
as informações atuais de metereologia, estas informações (que vem em um formato
chamadao METAR) são processadas pelo backend e mostradas ao usuário de uma forma 
fácil de entender.


