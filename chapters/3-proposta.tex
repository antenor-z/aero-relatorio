% Revisão OK 26/09
\chapter{A Proposta}
A ideia do trabalho é unir as funcionalidades do METAR-TAF com o AISWEB em uma 
aplicação web que o usuário, mesmo que leigo em aviação, consiga usar sem 
dificuldades.

Pelo fato da aviação necessitar ser um ambiente seguro e bastante regulado, considerando
que meu projeto é apenas um protótipo, prefiro restringir o caso de uso apenas
para jogadores de simuladores de voo que desejam que a simulação seja parecida
com o real. As páginas terão um aviso \textbf{"APENAS PARA USO EM SIMULADOR"}.

Dito isto, o sistema possui backend escrito na linguagem Python, fazendo uso da 
framework FastAPI. Até a entrega do Projeto I, o backend era feito com Flask. 
A escolha foi feita pela minha familiaridade com a framework. Porém, depois, 
descobri a framework FastAPI no meu trabalho. Normalmente, ela não é usada para 
fullstack, mas sim para fazer APIs REST, que retornam JSON na resposta de uma 
requisição. Um frontend feito em alguma framework de JavaScript faria a requisição
 e atualizaria a página a partir dos dados recebidos.

Contudo, é possível retornar qualquer tipo de dado no FastAPI, inclusive páginas 
HTML. Como prefiro fazer a renderização de páginas no lado do servidor, continuei
usando a funcionalidade de templates do Jinja2.

A substituição da framework deu-se pelos seguintes motivos:

\begin{itemize}
\item Projeto mais maduro, continuamente mantido;
\item Documentação bem mais detalhada do que a do Flask, com vários tutoriais sobre assuntos comumente usados, como autenticação e validação de dados;
\item Pensado para validação de dados, se integra muito bem ao Pydantic;
\item Servidor embutido (Uvicorn) extremamente fácil de configurar e suficiente para produção \cite{fast-api-prod}.
No Flask é necessário configurar alguns parâmetros para o Guinicorn via o arquivo "gunicorn\_config.py";
\item Assíncrono, permitindo o uso junto com bibliotecas que utilizam asyncio;
No Flask, para paralelismo, é necessário usar uma biblioteca como o Gevent.
\end{itemize}

No segundo semestre de 2023 comecei a fazer este projeto para uso próprio.
O código está disponível em \url{https://github.com/antenor-z/aero}. Atualmente o
projeto funciona, mas a arquitetura foi feita sem muito planejamento, as
informações do aeroporto fazem parte do código.

O usuário tem acesso a informações de frequência da torre, solo, tráfego, rampa
e operações, bem como das frequências e dados para VOR (um sistema de radionavegação
por antenas no solo), ILS (sistema de pouso por instrumentos) e informações de 
pista. Neste trabalho quero, armazenar estas em um banco de dados relacional com 
uma arquitetura bem planejada. Farei testes de desempenho simulando uma alta taxa 
de acesso.

Os aeródromos podem, ao longo do tempo, mudarem alguma frequência e outras
informações, como o número da pista, que muda a depender da variação do norte 
magnético, uma ampliação da pista faz a informação de comprimento precisar ser 
alterada...

O código precisava ser alterado para atualizar estas informações. Será implementado
 um sistema administrativo no site, com uma autenticação por senha e TOTP, para 
 que seja possível mudar qualquer informação no banco mesmo por pessoas que não 
 saibam programar.

Esta parte administrativa, possui controle de permissões. Ao criar uma conta, é
 informado quais aeroportos esta conta pode alterar. Também pode ser informado que
  esta é uma conta "super" que pode alterar todos os aeroportos e também criar e 
  apagar aeroportos.
