\chapter{Decodificação do METAR}
\subsection{Introdução}
É importante saber a direção do vento para operações de decolagem e pouso,
pois o vento pode ajudar ou atrapalhar nestas fases críticas de voo.
O melhor cenário possível é vento vindo de frente (vento de proa) pois
será necessário menos velocidade em relação ao solo para decolagem ou 
pouso. Para efeitos de sustentação da asa do avião e o que é mostrado
no velocímetro é a velocidade em relação ao ar. A velocidade do avião 
em relação do solo para o vetor do vento paralelo ao vetor velocidade será:

\begin{equation}
    V_solo = V_aviao + V_vento
    V_aviao = V_solo - V_vento
\end{equation}

No caso anterior do vento estar contra o avião, a velocidade do vento
fica negativa, portanto a velociade em relação ao ar fica maior para
uma mesma velociade com o solo.

No caso oposto, com o vento de cauda, o avião precisa manter uma
velociade em relação ao solo maior para manter a sustentação. No
pouso, o que importa é a velociade
no solo, pois agora é necessário desacelerar na pista. Portanto,
um vento de cauda aumento o comprimento de pista necessário para a 
parada da aeronave.

Contudo, normalmente o vento nunca estará perfeitamente paralelo
com a aeronave: é preciso calcular a componente do vetor vento paralela 
e perpendicular à aeronave. O vetor vento aponta para a direção que o vento sopra.

Sendo V o vetor do vento, \theta o menor ângulo entre o vetor 
do vento e o vetor com direção da proa da aeronave.


\begin{equation}
V = (V_x, V_y)
V_paralelo = V * cos(\theta)
V_perpendicular = V * sen(\theta)
\end{equation}

