\chapter{Decodificação do TAF}

O TAF (Terminal Aerodrome Forecast) é uma informação meteorológica que fornece
previsões para aeroportos, sendo uma ferramenta crucial para a aviação. Diferente
do METAR, que relata as condições meteorológicas atuais, o TAF projeta o que se
espera para as próximas horas e/ou dias. Ele é composto por diversas seções,
cada uma contendo informações específicas sobre as condições previstas para o
período. Uma diferença em relação ao METAR é que o TAF se entende por várias linhas.
A partir da segunda linhas temos os grupos temporais. Tudo que está em uma linha
são condições que irão ocorrer dentro de uma faixa de tempo. 

Neste capítulo, será apresentado um exemplo de TAF, sua decodificação,
e uma análise da complexidade temporal do algoritmo utilizado.

\section{Exemplo}
O TAF emitido para o aeroporto de Tancredo Neves (SBCF) no dia 21 de abril de 2024 às 07:00Z foi:

\begin{verbatim}
TAF SBCF 210700Z 2112/2212 00000KT 7000 NSC TX31/2119Z TN15/2209Z 
  BECMG 2112/2114 11005KT 
  PROB30 2117/2121 04005KT 
  BECMG 2201/2203 00000KT RMK PGF
\end{verbatim}

Vamos decodificar este TAF passo a passo:

Data e Hora de Emissão:
\texttt{210905Z} indica que o TAF foi emitido no dia 21 às 09:05Z.

Período de Validade:
\texttt{2112/2124} indica que o TAF é válido do dia 21 às 12:00Z até o dia 21 às 24:00Z.

Condições Iniciais:
\texttt{34005KT 7000 SCT015} descreve as condições previstas para o início
do período de validade. \texttt{34005KT} significa que o vento estará vindo
da direção 340° com uma velocidade de 5 nós. \texttt{7000} indica visibilidade
de 7 km, e \texttt{SCT015} significa que haverá nuvens dispersas a 1500 pés.

Temperatura Mínima e Máxima:
\texttt{TN21/2112Z TX25/2116Z} indica que a temperatura mínima prevista é de 21°C
às 12:00Z do dia 21, e a máxima será de 25°C às 16:00Z do dia 21.

A partir de agora começam os grupos separados por linha que comentei na introdução deste
capítulo.

Mudança Prevista (\texttt{BECMG}):
\texttt{BECMG 2113/2115 9999 FEW015} indica que entre 13:00Z e 15:00Z do dia 21,
a visibilidade aumentará para 10 km ou mais (\texttt{9999}), e haverá poucas nuvens
a 1500 pés (\texttt{FEW015}).

Nova Mudança Prevista:
\texttt{BECMG 2116/2118 16010KT CAVOK} sugere que entre 16:00Z e 18:00Z do dia 21,
o vento mudará para 160° com uma velocidade de 10 nós, e as condições se tornarão
"CAVOK" (Ceiling and Visibility OK), indicando que a visibilidade e o teto de nuvens
estão dentro dos limites favoráveis para operações de voo.

Observação:
\texttt{RMK PHD} adiciona uma observação, é algo específico dos TAFs brasileiros.
O trigrama PHD representa a indentificação do previsor que gerou este TAF \cite{trigrama-taf}.

\section{Algoritmo}

O algoritmo para lidar com um TAF é semelhante ao utilizado para o METAR, porém adaptado
para as especificidades das previsões temporais. O TAF é segmentado em diversas
partes, cada uma sendo processada individualmente por expressões regulares para extrair as
informações relevantes.

Para cada item do TAF, o algoritmo tenta associar a string correspondente a uma descrição
detalhada utilizando expressões regulares pré-definidas. Por exemplo, a expressão
\texttt{\d4/\d4\d4/\d4} é usada para identificar os períodos de tempo, enquanto a expressão
\texttt{(\d{3})(\d{2})KT} é utilizada para extrair a direção e velocidade do vento.

Assim como no METAR, as informações são organizadas em tuplas que contêm a string original
e sua decodificação correspondente. Essas tuplas são então enviadas para um sistema de
templating Jinja2, para a geração da página HTML.

\section{Complexidade Temporal}

A complexidade temporal da decodificação de um TAF segue um padrão similar ao do METAR.
A função \texttt{re.findall()} é utilizada para identificar e extrair informações, e a
complexidade temporal desta operação depende do número de caracteres da expressão regular
(m) e da string do TAF a ser analisada (n).

Considerando que as expressões regulares utilizadas no TAF também são simples, a complexidade
temporal de decodificação de um TAF pode ser representada por:

$$ O_{decode_TAF}(p * n)$$

Onde:

\begin{verbatim}
p := Quantidade de itens do TAF.
n := Quantidade de caracteres da string do TAF.
\end{verbatim}

Como no caso do METAR, as variáveis "p" e "n" são as que influenciam diretamente a
complexidade do algoritmo.