\chapter{Principios norteadores}

Para este projeto segui os seguintes princípios por ordem de prioridade.

\begin {enumerate}
\item Corretude da informação apesentada
\item Rapidez de carregamento das páginas
\item Facilidade de uso do sistema
\end {enumerate}

\section{Corretude da informação apesentada}
Não podem haver erros na explicação do METAR, TAF e nas informações do aeródromo.

\section{Rapidez de carregamento das páginas}
Não são coletados externamente nenhum
arquivo como .js, .css, .ttf, etc. Os arquivos estáticos são se encontram no servidor.
A geração de página é feita no lado do servidor. O que
precisa rodar no lado do cliente como pesquisa e tooltips são implementadas com JavaScript puro.

Os plots com informações históricas são construídos assincronamente, os dados de METAR e TAF
são coletados da API do Aviation Weather e inseridos no banco também assincronamente. então
quando o usuário carrega a página estas informações já estão prontas para o envio.

\section{Facilidade de uso do sistema}
A diagramação das páginas e feita tendo em mente que um usuário que não possui um conhecimento
avançado em aviação, mas deseja saber todas as informações de um aeródromo. Porém estas
informações são separadas em três páginas para cada aeródromo para que não fique visualmente
sobrecarregado.

Por exemplo, para o aeroporto do Galeão no Rio de Janeiro, temos as paginas:

\begin{itemize}
\item https://aero.a4barros.com/info/SBGL: Informações de pista, frequências e METAR explicado
\item https://aero.a4barros.com/taf/SBGL: TAF explicado
\item https://aero.a4barros.com/info/SBGL/history: Plots com informações históricas
\end{itemize}

