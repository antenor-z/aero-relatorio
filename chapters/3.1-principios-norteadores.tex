% Revisão OK 25/09
\chapter{Princípios norteadores}

Para este projeto segue-se os seguintes princípios por ordem de prioridade começando
pela maior prioridade.

\section{Exatidão da Informação Apresentada}
Não deve haver erros na explicação do METAR, TAF e nas informações do aeródromo. 
Claro que não é possível aderir a este princípio em 100\% dos casos, pois uma 
informação de aeroporto pode mudar. No entanto, é comparada constantemente as 
informações locais com o AISWeb e, em caso de alterações, é possível alterá-las 
facilmente pela área restrita do site. Normalmente, estas informações não 
mudam com tanta frequência. Um aeroporto pode passar anos ou até décadas sem
nenhum dado ser alterado.

\section{Rapidez no Carregamento das Páginas} Nenhum arquivo externo como .js, 
.css, .ttf, etc., é coletado externamente. Os arquivos estáticos estão no servidor. 
A geração das páginas, inclusive dos gráficos, é feita no lado do servidor. O que 
precisa rodar no lado do cliente, como pesquisas e \textit{tooltips}, é implementado com 
JavaScript puro.

Os gráficos com informações históricas são construídos de forma assíncrona, e os
dados de METAR e TAF são coletados da \textit{API} do \textit{Aviation Weather} e inseridos no
banco de dados também de forma assíncrona. Assim, quando o usuário carrega a 
página, essas informações já estão prontas para o envio.

Como será visto no capítulo de arquitetura existe uma proxy reversa com NGINX,
esta proxy faz cache das páginas, então se uma página é recarregada muitas vezes
seguidas o NGINX entrega a página salva no cache. O cache está configurado para
ser invalidado em um minuto, então, no pior caso, as informações ficam com um 
minuto de atraso. A informação que atualiza com mais frequência é o METAR. A
atualização ocorre de hora em hora, então o atraso de um minuto, no máximo, é
aceitável.

\section{Facilidade de Uso do Sistema} A diagramação das páginas é feita
considerando que o usuário pode não ter um conhecimento avançado em aviação, mas
deseja acessar todas as informações de um aeródromo. Essas informações estão 
divididas em três páginas para cada aeródromo, para que a visualização não fique 
sobrecarregada.