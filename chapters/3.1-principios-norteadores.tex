\chapter{Principios norteadores}

Para este projeto segui os seguintes princípios por ordem de prioridade começando pela maior
prioridade.

\section{Correção da Informação Apresentada}
Não devem haver erros na explicação do METAR, TAF e nas informações do aeródromo. Claro que não
é possível aderir a este princípio em 100\% dos casos, pois uma informação de aeroporto pode mudar.
No entanto, comparo constantemente minhas informações com o AISWeb e, em caso de alterações,
é possível fazê-las facilmente pela área restrita do site.

\section{Rapidez no Carregamento das Páginas} Nenhum arquivo externo como .js, .css, .ttf, etc., 
é coletado externamente. Os arquivos estáticos estão no servidor. A geração das páginas inclusive dos
gráficos é feita
no lado do servidor.
O que precisa rodar no lado do cliente, como pesquisas e tooltips, é
implementado com JavaScript puro.

Os gráficos com informações históricas são construídos de forma assíncrona, e os dados de METAR
e TAF são coletados da API do Aviation Weather e inseridos no banco de dados também de forma
assíncrona. Assim, quando o usuário carrega a página, essas informações já estão prontas para o envio.

\section{Facilidade de Uso do Sistema} A diagramação das páginas é feita considerando que o usuário
pode não ter um conhecimento avançado em aviação, mas deseja acessar todas as informações de um
aeródromo. Essas informações estão divididas em três páginas para cada aeródromo, para que a
visualização não fique sobrecarregada.

Por exemplo, para o aeroporto do Galeão no Rio de Janeiro, temos as seguintes páginas:

\begin{itemize}
    \item https://aero.a4barros.com/info/SBGL: Informações de pista, frequências e METAR explicado
    \item https://aero.a4barros.com/taf/SBGL: TAF explicado
    \item https://aero.a4barros.com/info/SBGL/history: Gráficos com informações históricas
\end{itemize}