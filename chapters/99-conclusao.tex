\chapter{Conclusão e Próximos Passos}

O desenvolvimento deste projeto busca juntar as funcionalidades comumente usadas na 
simulação de voo em uma plataforma acessível e amigável, buscando promover um 
maior nível de realismo na simulação de voo, aspecto desejado por quem leva
a simulação de voo "a sério".

A arquitetura implementada, utilizando Docker e Docker Compose, garante uma 
implementação modular, segura e, com o uso do Git, de fácil deploy caso seja 
necessário trocar o ISP no futuro. O uso de um banco de dados como
o MariaDB e da funcionalidade Docker Secrets para armazenar as senhas, proporciona um 
ambiente robusto para o projeto.
O uso de uma VPS modesta para hospedagem do projeto em produção demonstra a 
viabilidade do sistema em ambientes com recursos limitados.

Para o próximo semestre está planejado implementar o 
módulo que informa a pista ativa. É mais seguro para o voo se a decolagem for 
realizada a partir da cabeceira em que o vento está soprando contra o sentido 
da movimentação do avião. Por isso, em um momento os procedimentos estão sendo 
realizados em um lado da pista e em outro momento pelo outro lado.

Nem sempre a direção do vento estará paralela com a pista, mas, mesmo assim, 
tenta-se decolar e pousar no sentido que a componente paralela ao sentido do 
movimento da aeronave fique com sentido contrário.

A pista ativa normalmente é informada pela torre de controle, mas é 
interessante o piloto se antecipar.

Outro módulo que será implementado é o cálculo do perfil de descida. Para que 
o avião chegue no procedimento de aproximação com a altitude expressa na 
carta de aproximação, é necessário calcular uma razão de descida em pés por 
minuto, a partir da altitude atual, da altitude que se quer chegar e da 
velocidade atual.

É interessante poder adicionar e editar aeródromos sem precisar estar logado
na máquina de produção via SSH. Será feito uma seção do site protegida por
nome de usuário e senha que disponibilizará uma interface para estas mudanças.