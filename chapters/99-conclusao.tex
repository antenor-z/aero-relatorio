% Revisão OK 12/10
\chapter{Conclusão}

Como conclusão digo que foi um projeto muito interessante de se fazer já que 
gosto tanto da área de programação para web como a de aviação. É o meu primeiro 
projeto open-source grande e os utilizadores podem melhorar o produto mesmo sendo
leigos em programação já que na parte inferior da página principal há um e-mail 
de contato para que me sejam enviadas elogios, críticas e sugestões.

Explorei também um pouco do frontend, área que não tinha tanta familiaridade.

Acredito que o objetivo do projeto foi alcançado. O Aero busca juntar as funcionalidades 
comumente usadas na simulação de voo em uma plataforma acessível e amigável, 
trazendo um maior nível de realismo na simulação de voo, aspecto desejado por quem leva
a simulação de voo "a sério".

Vários assuntos que não conhecia aprendi com este projeto incluindo:

\begin{itemize}
\item Fazer uma página de observabilidade
\item Fazer um teste de carga
\item Usar o Docker Secrets para armazenar senhas ao invés do ".env"
\item Usar cache direto no NGINX para evitar que o site saia do ar em períodos 
de muitos acessos
\item Armazenar hash de senha com salt o padrão bcrypt
\item Implementar autenticação de dois fatores TOTP
\item Traduzir mensagens de erros do Pydantic para o Português para que possam 
ser mostrados para o usuário.
\end{itemize}

Acredito fortemente que este foi um projeto muito proveitoso devido aos 
conhecimentos de programação web que adquiri que serão de grande valia no meu 
trabalho que é o desenvolvimento do backend de um PWA. Além disso, já que o site está público 
servirá como uma vitrine para outras potenciais oportunidades de carreira.

\section {Código completo}

O código fonte na integra do Aero pode ser visto acessando o seguinte link: \url{https://github.com/antenor-z/aero}

O código LaTeX que gerou este documento está disponível em: \url{https://github.com/antenor-z/aero-relatorio}

Por fim, o código apresentação de slides feita no Beamer está disponível em: \url{https://github.com/antenor-z/aero-presentation}