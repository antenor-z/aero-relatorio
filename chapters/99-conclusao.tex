\chapter{Conclusão e Próximos Passos}

Como conclusão digo que foi um projeto muito interessante de se fazer já que 
gosto tanto da área de programação para web como a de aviação. É o meu primeiro 
projeto open-source grande e os utilizadores podem melhorar o produto mesmo não 
sabendo de programação já que na parte inferior da página principal há um e-mail 
de contato para que me sejam enviadas elogios, críticas e sugestões.

O desenvolvimento deste projeto buscou juntar as funcionalidades comumente usadas 
na simulação de voo em uma plataforma acessível e amigável, buscando promover um 
maior nível de realismo na simulação de voo, aspecto desejado por quem leva
a simulação de voo "a sério".

Vários assuntos que não conhecia aprendi com este projeto incluindo:

\begin{itemize}
\item Fazer uma página de observabilidade
\item Fazer um teste de carga
\item Usar cache direto no NGINX para evitar que o site saia do ar em períodos 
de muitos acessos
\item Implementar hash de senha com salt usando o biblioteca bcrypt
\item Usar autenticação de dois fatores TOTP
\item Traduzir mensagens de erros do Pydantic para o Português para que possam 
ser mostrados para o usuário.
\end{itemize}

Acredito fortemente que este foi um projeto muito proveitoso devido aos 
conhecimentos de programação web que adquiri que serão de grande valia no meu 
trabalho que também é de desenvolvimento web, mas, já que o site está público 
servirá como uma vitrine para outras potenciais oportunidades de carreira.