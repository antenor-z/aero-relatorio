% Revisão OK 30/09
\chapter{Disponibilidade do METAR}

Durante as enchentes de grandes proporções no Rio Grande do Sul ocorridas em maio 
de 2024, grande parte da capital Porto Alegre ficou alagada. O Aeroporto 
Internacional Salgado Filho fechou. Já que grande parte do pátio e da pista 
ficaram debaixo de água estavam impossibilitados pousos e decolagens. Algumas 
aeronaves particulares foram danificadas pela água.

Com o aeroporto fechado, o METAR não era mais emitido. Porém, a API do Aviation 
Weather não respondia com um código HTTP de erro ou de não encontrado, mas sim 
com código de sucesso (200) e uma string vazia como resposta. Com isso, a decodificação
do METAR falhava. Para este aeroporto, a seguinte tela aparecia:

\begin{figure}[ht]
    \begin{center}
    \includegraphics[width=400pt]{img/sbpa-erro.jpeg}
    \caption{Página com o erro}
    \label{fig:sbpa-erro}
    \end{center}
\end{figure}

Quando o módulo de decodificação do METAR tentava buscar o dia e hora na string 
do METAR, esta não existia. Então, quando tentava buscar o índice nos grupos 
de captura, este não existia. Então a exceção IndexError era levantada.

Já que todo METAR deve possuir o item com o dia e a hora zulu, caso este não 
exista, a função de decodificação do METAR retorna o seguinte valor:

\begin{verbatim}
[(" ", "METAR indisponível.")]
\end{verbatim}

Este valor consegue ser interpretado pelo template.

\begin{figure}[ht]
    \begin{center}
    \includegraphics[width=400pt]{img/sbpa.jpeg}
    \caption{Página com o fallback}
    \label{fig:sbpa}
    \end{center}
\end{figure}
