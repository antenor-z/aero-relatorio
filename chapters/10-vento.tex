\chapter{Decodificação do METAR}

Primeiramente, em aviação, falamos de duas velocidades: a velocidade do avião
 em relação a um ponto fixo no solo é a velocidade de solo (ground speed).
Com ela podemos calcular quanto tempo uma viagem demora. Mas, para efeitos de 
sustentação do avião e o que importa é a velocidade em relação ao ar (air speed). 

É importante saber a direção do vento para operações de decolagem e pouso,
pois o vento pode ajudar ou atrapalhar nestas fases críticas de voo.
O melhor cenário possível é vento vindo de frente (vento de proa) pois
será necessário menos velocidade em relação ao solo para decolagem ou 
pouso.

A velocidade do avião em relação do solo no caso do vetor vento 
paralelo ao vetor velocidade será:

\begin{align}
    V_{solo} = V_{ar} + V_{vento} \\
    V_{ar} = V_{solo} - V_{vento}
\end{align}

Sendo $V_{aviao}$ definido com sempre positivo, se o vento está 
contra o avião, o sinal de $V_{vento}$ do vento
será negativo, portanto a velociade em relação ao ar fica maior para
uma mesma velociade com o solo.

No caso oposto, com o vento de cauda, o avião precisa manter uma
velocidade em relação ao solo maior para compensar. Para o 
pouso será necessária mais distância para a parada da aeronave.
Já para a decolagem, o avião terá que usar mais pista até que
consiga levantar voo.

Contudo, normalmente o vento nunca estará perfeitamente paralelo
com a aeronave: é preciso calcular a componente do vetor vento 
paralela e perpendicular à aeronave.

Sendo V o vetor do vento, $\theta$ o menor ângulo entre o vetor 
do vento e o vetor com direção da proa da aeronave.


\begin{equation}
V = (V_x, V_y)
V_paralelo = V * cos(\theta)
V_perpendicular = V * sen(\theta)
\end{equation}

