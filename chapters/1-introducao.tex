% Revisão OK 25/09
\chapter{Introdução}
Com o aumento da capacidade de passageiros e carga e a necessidade 
de uma maior segurança, começou a se fazer necessário trazer ao cockpit 
vários documentos como checklist de procedimentos; log book; cartas de 
navegação, de saída, de aproximação, do aeródromo, tabelas de performance
 da aeronave etc.

Para levar tudo isto costumava-se usar uma maleta (a Flight Bag). Obviamente 
esta ficava muito pesada.

Com a miniaturização dos computadores e surgimento dos tablets, começaram a ser 
desenvolvidos programas que substituem partes ou todos estes documentos, é a
chamada maleta de voo eletrônica, mais conhecida pela sigla em Inglês EFB 
(\textit{Electronic Flight Bag}).

Atualmente existem hardwares dedicados para esta função, mas é mais comum se 
usar um tablet com um aplicativo disponibilizado pela companhia aérea. Normalmente, 
o tablet escolhido é um iPad da Apple, mas algumas companhias optaram pelo 
Microsoft Surface. \cite{surface}

O uso do EFB trouxe uma série de benefícios para os pilotos e para as companhias
aéreas. Além de reduzir o peso e o volume de documentos físicos a serem 
transportados, o EFB permite uma rápida atualização das informações, garantindo
 que os pilotos tenham sempre acesso às versões mais recentes das cartas de 
 navegação. \cite{EFB-more-than}

Além disso, o EFB possibilita o acesso a uma vasta quantidade de informações 
adicionais, como manuais de operação da aeronave, cálculo de consumo de combustível,
cálculo de velocidades de decolagem e dados meteorológicos em tempo real, o que 
contribui para uma tomada de decisão mais informada e segura durante o voo.
