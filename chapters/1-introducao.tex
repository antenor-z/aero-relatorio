\chapter{Introdução}
Com o aumento da capacidade de passageiros e carga e a 
necessidade de uma maior 
segurança, começou a se fazer necessário trazer ao cockpit vários
documentos como checklist de procedimentos; log book; cartas de 
navegação, de saída, de aproximação, do aeródromo; tabelas de 
performance da aeronave etc. 

Para levar tudo isto costumava-se usar uma maleta (a Flight Bag),
obviamente esta ficava muito pesada.

Com a miniaturização dos computadores e surgimentos dos tablets, 
começaram a ser desenvolvidos programas que substituíam partes
ou todos estes documentos, é a chamada maleta de voo eletrônica,
mais conhecida pela sigla em Inglês EFB (\textit{Electronic Flight Bag}).

Atualmente existem hardware dedicados para esta função, mas é
mais comum se usar um tablet com um aplicativo disponibilizado pela
companhia aérea.