\chapter{Front-end}

Como falado no capítulo da arquitetura, a página é gerada server-side, então o que 
é retornado para cada rota é um HTML já pronto. Acredito que, para o meu caso, é mais 
performático fazer assim do que usar páginas com Javascript que fazem requisição para uma API REST.

De todo modo, no final da execução de uma rota, um dicionário Python é gerado, algo que 
poderia ser facilmente convertido para um JSON usando a função \texttt{dumps()} da 
biblioteca \texttt{json} do próprio Python. No caso desta aplicação, este dicionário é 
enviado para o template usando a função \texttt{render\_template()} da biblioteca Jinja2, 
que recebe o nome da página HTML com o template e um número qualquer de \textit{kwargs} (argumentos nomeados) 
que podem ter qualquer tipo serializável, incluindo dicionários.

Perceba que não há problema de acoplamento fazendo deste modo, pois não há código HTML sendo 
escrito dentro do backend. Como já dito, o que é passado para o Jinja é algo equivalente a JSON.

Um página template é um arquivo HTML com \textit{placeholders} que serão substituídos pelos 
\textit{kwargs} de mesmo nome. O Jinja2 tem estruturas de repetição para que um código HTML 
possa ser repetido usando valores da lista. E, no caso de dicionários, é fácil acessar 
os valores. Neste projeto, para exibir a lista de frequência, o seguinte código é usado.

\lstinputlisting[label=cod:jinja-comm-old,title={Formatação Antiga},caption={Formatação Antiga},language=SQL]{code/jinja-comm-old.html}

Note que é possível fazer operações e formatações simples no Jinja2. Já que a frequência 
é armazenada no banco como um inteiro de ponto fixo (como dito no capítulo de modelo de dados), 
aqui eu poderia dividir o valor por mil e formatar com três casas decimais para que, caso 
uma frequência termine com zeros à direita, sempre tenhamos três dígitos decimais, que 
é o padrão para frequências de comunicação.

Porém, para não misturar a \textit{view} com o \textit{controller}, deixei este processamento 
de dados para uma função dentro do controlador.

\lstinputlisting[label=cod:jinja-comm,title={Formatação Atual},caption={Formatação Atual},language=SQL]{code/jinja-comm.html}

No código do projeto, na pasta 'templates', é possível ver todos os templates usados.
