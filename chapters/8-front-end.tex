\chapter{Front-end}

Como falado no capítulo da arquitetura, a página é gerada server-side, então o que 
é retornado para cada rota é um HTML já pronto. Acredito que, para o meu caso, é mais 
performático fazer assim do que usar páginas com Javascript que fazem requisição para uma API REST.

De todo modo, no final da execução de uma rota, um dicionário Python é gerado, algo que 
poderia ser facilmente convertido para um JSON usando a função \texttt{dumps()} da 
biblioteca \texttt{json} do próprio Python. No caso desta aplicação, este dicionário é 
enviado para o template usando a função \texttt{render\_template()} da biblioteca Jinja2, 
que recebe o nome da página HTML com o template e um número qualquer de \textit{kwargs} (argumentos nomeados) 
que podem ter qualquer tipo serializável, incluindo dicionários.

Perceba que não há problema de acoplamento fazendo deste modo, pois não há código HTML sendo 
escrito dentro do backend. Como já dito, o que é passado para o Jinja é algo equivalente a JSON.

Um página template é um arquivo HTML com \textit{placeholders} que serão substituídos pelos 
\textit{kwargs} de mesmo nome. O Jinja2 tem estruturas de repetição para que um código HTML 
possa ser repetido usando valores da lista. E, no caso de dicionários, é fácil acessar 
os valores. Neste projeto, para exibir a lista de frequência, o seguinte código é usado.

Note que é possível fazer operações e formatações simples no Jinja2. Já que a frequência 
é armazenada no banco como um inteiro de ponto fixo (como dito no capítulo de modelo de dados), 
foi criado o filtro "frequency3" para exibir o número corretamente. Um "filtro" no Jinja 
é apenas uma função que recebe e retorna
uma string. O filtro é "chamado" usando a sintaxe "\{\{variavel | funcao\}\}". Para os que tem
experiência com Linux é parecido com a ideia do operador "pipe".

\lstinputlisting[label=cod:jinja-comm,title={Template de comunicação com o Jinja},caption={Template de comunicação com o Jinja},language=SQL]{code/jinja-comm.html}

Note que caso a variável "isAdmin" seja definida, um botão de alterar a frequência aparece. Este
e outros botões de adição e edição são mostrados quando o login é feito para que o administrador
consiga editar um aeródromo.

No código do projeto, na pasta 'templates', é possível ver todos os templates usados.

\section{Minificação}

Removendo os espaços e caracteres de nova linha é possível diminuir o tamanho dos arquivos enviados
para o usuário. Após este processo, cada arquivo html, js etc fica com apenas uma linha, os menos linhas
no caso dos css, obviamente não é fácil para um humano entender, mas o navegador consegue fazer o 
parsing sem problemas. Isto é chamado de {\em minification} ou minificação. Com os arquivos
com menor tamanho, a transferência servidor para cliente é terminada mais rápido, logo a experiência
para o usuário torna-se mais agradável, já que as páginas carregam mais rápido.

Para a tabela abaixo cada tamanho em kB e cada tempo em ms se refere a média simples dos valores 
encontrados na aba "rede" das ferramentas de desenvolvedor do navegador Firefox em cinco 
carregamentos da página. A opção de desabilitar cache foi usada.

Legenda para a tabela abaixo
\begin{itemize}
\item \textbf{A:} Tamanho do arquivo sem minification
\item \textbf{B:} Tempo de carregamento deste arquivo
\item \textbf{C:} Tamanho do arquivo com minification
\item \textbf{D:} Tempo de carregamento deste arquivo
\end {itemize}

O "Tamanho" refere-se a quantidade de bytes transferidos (com compactação gzip) e não ao 
tamanho do arquivo após a compactação já que isto é feito no lado do usuário pelo navegador.

\begin{longtable}{|p{4cm}|p{2.15cm}|p{2.15cm}|p{2.15cm}|p{2.15cm}|}
    \caption{Com e Sem minification} \\
    \hline
    \textbf{Arquivo} & \textbf{A} & \textbf{B} & \textbf{C} & \textbf{D} \\ \hline
    \endfirsthead
    \multicolumn{5}{c}%
    {{\tablename\ \thetable{} -- Continuação da página anterior}} \\
    \hline
    \textbf{Arquivo} & \textbf{A} & \textbf{B} & \textbf{C} & \textbf{D} \\ \hline
    \endhead
    \hline \multicolumn{5}{|r|}{{Continua na próxima página}} \\ \hline
    \endfoot
    \hline
    \endlastfoot
        SBGR
        & 5,88 kB
        & 130 ms
        & 4,93 kB
        & 57 ms
        \\ \hline
        style.css
        & 3,45 kB
        & 68 ms
        & 2,49 kB
        & 33 ms
        \\ \hline
        tooltip.css
        & 960 B
        & 67 ms
        & 831 B
        & 24 ms
        \\ \hline
        rwy.css
        & 590 B
        & 36 ms
        & 518 B
        & 38 ms
        \\ \hline
        \texttt{Total}
        & 11.35 kB
        & 363 ms
        & 9.21 kB
        & 196 ms
        \\ \hline
\end{longtable}

Fazendo

\[
\text{x\%} = \frac{x_{\text{initial}} - x_{\text{final}}}{x_{\text{initial}}} \times 100\%
\]

É possível ver uma economia de 18,9\% na quantidade de informações enviadas. E um tempo de resposta
46,0\% menor.

