% Revisão OK 30/09
\chapter{Cronograma}

Para ter uma direção do projeto e medição da evolução, ele foi dividido em várias 
tarefas com início e duração esquematizados abaixo.

O que está em laranja são grupos de tarefas, em preto são as tarefas em si, as 
células com fundo em azul são a(s) semana(s) planejadas para realizar cada tarefa. 
O símbolo de losango mostra quando a tarefa foi de fato feita.

Em vermelho, temos tarefas que não estavam no planejamento.

\section{Cronograma do Projeto Final I}

\begin{figure}[ht]
    \begin{center}
    \includesvg[width=\linewidth]{img/cronograma.svg}
    \caption{Cronograma I}
    \label{fig:cronograma-planejado-I}
    \end{center}
\end{figure}

Os dias 29/05 e 26/06 são milestones onde ocorrem as entregas.

\begin{table}[h]
    \centering
    \caption{Milestones}
    \begin{tabular}{|c|l|}
        \hline
        \textbf{Data} & \textbf{Milestone} \\
        \hline
        29/05 & Entrega da proposta de Projeto Final I \\
        26/06 & Entrega do relatório de Projeto Final I \\
        \hline
    \end{tabular}
\end{table}

Por ter sido meu primeiro grande projeto autogerido, tive um pouco de dificuldade 
em estimar o tempo real de implementação de cada tarefa. Portanto, há uma grande 
diferença entre quando uma tarefa foi estimada para ser feita e quando ela realmente 
o foi.

As seguintes tarefas são propostas para a segunda parte do projeto.

\begin{itemize}
    \item Usuários autenticados alterarem informações de um aeroporto;
    \item Cálculo das componentes do vento;
    \item Cálculo do perfil de descida;
    \item Geração de gráficos históricos para um aeroporto;
    \item Visualização das posições dos aeroportos em um mapa.
\end{itemize}

\section{Cronograma do Projeto Final II}

\begin{figure}[ht]
    \begin{center}
    \includesvg[width=\linewidth]{img/cronograma2.svg}
    \caption{Cronograma II}
    \label{fig:cronograma-planejado-II}
    \end{center}
\end{figure}

\begin{table}[h]
    \centering
    \caption{Milestones}
    \begin{tabular}{|c|l|}
        \hline
        \textbf{Data} & \textbf{Milestone} \\
        \hline
        06/09 & Entrega do formulário de Projeto Final II \\
        15/11 & Entrega do Projeto Final II \\
        25/11 a 29/11 & Bancas Projeto Final II \\
        \hline
    \end{tabular}
\end{table}