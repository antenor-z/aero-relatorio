\chapter{Decodificação do METAR}

\section{Introdução}
O METAR é um protocolo de transmissão de dados meteorológicos de um aeroporto ou aeródromo. Não se trata de 
uma previsão do tempo, mas sim de uma visualização atual. O METAR é formado por itens separados por espaço. 
Cada item corresponde a uma unidade mínima de informação meteorológica. Com os dados de sensores instalados 
no aeródromo \cite{metar-weather-gov}, a cada hora é publicado um novo METAR que é válido para aquela hora. 
Em casos excepcionais, quando as condições de tempo estiverem mudando repentinamente, um METAR pode ser atualizado 
a cada meia hora \cite{METAR-speci}.

\section{Exemplo}
O METAR no aeroporto de Fortaleza \cite{METAR-sbfz}, no dia 17 de abril de 2024 às 10:54 foi
\texttt{171300Z 15010KT 9999 BKN019 SCT025 FEW030TCU BKN100 30/25 Q1011}.

"SBFZ" se refere ao código ICAO (International Civil Aviation Organization) do aeroporto, não confundir 
com o código IATA (International Air Transport Association) que é formado por três letras. O aeroporto Pinto 
Martins possui o código IATA FOR, o Santos Dumont SDU e o Galeão GIG. O público geral parece conhecer mais este 
código, mas na aviação costuma-se usar mais o código ICAO, pois \textit{todos} os aeródromos possuem um, 
enquanto o IATA só é presente em aeroportos onde há processamento de bagagem \cite{iata-codes} \cite{icao-codes}.

O ICAO é formado por quatro letras em que a primeira é o prefixo da região. A América do Sul possui o prefixo "S", 
o Brasil possui o prefixo "SB", por isso que os aeroporto de Fortaleza, Santos Dumont e Galeão possuem os códigos 
SBFZ, SBRJ e SBGL, respectivamente. Países com muitos aeroportos, apenas uma letra, 
logo as três últimas letras ficam livres, podendo assim ter mais códigos para uso.

\texttt{171300Z} significa que este METAR se refere ao dia 17 às 13 horas e zero minuto zulu. Horário zulu é 
simplesmente o fuso horário da longitude de zero grau, chamado de hora UTC ou Coordinated Universal Time \cite{UTC}. 
Para que não haja confusão com os horários, a aviação internacionalmente usa o horário UTC. Este METAR será 
válido até às 13:59, quando será substituído pelo METAR iniciando com "SBFZ 171400Z".

Note que a seguinte expressão regular com três grupos de captura consegue extrair o dia, a hora e o minuto:

\begin{verbatim}
([0-9]{2})([0-9]{2})([0-9]{2})Z
\end{verbatim}

Com o METAR supracitado, os grupos de captura serão:

\begin{itemize}
\item Grupo 1 (dia): 17
\item Grupo 2 (hora): 13
\item Grupo 3 (minuto): 00
\end{itemize}

\texttt{15010KT} se refere à velocidade e direção do vento. Os três primeiros algarismos informam a direção, 
em graus, de onde o vento sopra, e os últimos dois algarismos informam a velocidade do vento em nós (milhas náuticas por hora). 
Neste caso, o vento vem da direção 150 graus com velocidade de dez nós. Com a expressão abaixo extraímos essas duas informações:

\begin{verbatim}
([0-9]{3})([0-9]{2})KT
\end{verbatim}

A informação de vento pode também conter a letra G (gust) para rajadas e a letra V em um item separado para o 
caso de haver variação de direção. Por exemplo, um METAR com os itens \texttt{10016G21KT 080V120} informa que 
há rajadas de até 21 kt e a direção do vento pode variar de 80 a 120 graus. Existem outros aeroportos que podem 
usar outras unidades para a velocidade do vento, mas no Brasil só é usado nós (kt). Para obter essas informações 
usamos o regex \texttt{([0-9]{3}[0-9]{2}G{[0-9]{2}})} e \texttt{([0-9]{3})V([0-9]{3})}.

\texttt{9999} significa visibilidade ilimitada (maior ou igual a 10 km). Se fosse 6000, a visibilidade seria de 
6 km. Por ser sempre quatro algarismos, o regex \texttt{([0-9]{4})} consegue capturar essa informação.

\texttt{30/25} Temperatura 30°C e ponto de orvalho 25°C. Caso a temperatura seja negativa, a letra M é adicionada 
antes do número. M2/M5 significa temperatura -2°C e ponto de orvalho -5°C \cite{METAR-help}.

\texttt{Q1012} O altímetro do avião deve ser referenciado para 1012 hectopascal. Também pode ser usada a unidade 
polegadas de mercúrio (mmHg), mas no Brasil esta não é usada no METAR.

\texttt{SCT025} Nuvens espalhadas (3/8 a 4/8 do céu com nuvens) em 2500 pés de altitude. 025 se refere ao nível 
de voo (Flight Level), que é a altitude acima do nível médio do mar com divisão exata por 100.

\texttt{FEW030TCU} Poucas nuvens (1/8 a 2/8 do céu com nuvens) em 3000 pés de altitude. O sufixo TCU significa 
que há nuvens convectivas significativas \cite{decea-mil}.

\texttt{BKN100} Nuvens broken (5/8 a 7/8 do céu com nuvens) em 10000 pés de altitude.

Existe também o tipo OVC (overcast) que se refere a totalmente encoberto.

\section{Algoritmo}

O objetivo do módulo de decoder é dar uma explicação semelhante a esta para qualquer tipo de METAR de aeroportos 
no Brasil. O módulo usa várias expressões regulares para decodificar uma grande quantidade de informações, porém 
não é exaustivo; foi dada preferência a fenômenos que podem ocorrer no Brasil \cite{decea-mil}.

O algoritmo deve separar a string do METAR pelo caractere de espaço. Para cada item separado, cada expressão regular 
é testada. Caso uma combinação ocorra, os grupos de captura são interpolados em uma string que explica aquele item. 

Semdo "\$1" o primeiro grupo de captura e "\$2" o segundo, se o item "27008G16KT" é encontrado pela expressão 
\begin{verbatim}
  ([0-9]\{3\})([0-9]\{2\})G([0-9]\{2\})KT
\end{verbatim}

o algorítmo interpola a frase:

\begin{verbatim}
  Vento \$1° com \$2 nós e rajadas de até \$3 nós
\end{verbatim}

com os grupos de captura do regex supracitado. Então será gerada uma tupla
\texttt{(27008G16KT, Vento 270° com 8 nós e rajadas de até 16 nós)}. O retorno do 
algorítmo será uma lista de tuplas que será enviada a ferramenta de templating
de página Jinja.

\section{Complexidade Temporal}
Considerando que todas as expressões usadas são simples, isto é, não levam a backtracking, 
a complexidade temporal para executar a função \texttt{re.findall()} do Python é

$$ O_{findall}(m + n) $$

\begin{verbatim}
m := quantidade de caracteres da expressão regex
n := quantidade de caracteres da string a ser analisada
\end{verbatim}

Se temos que testar todas as expressões para cada item do METAR, a complexidade será

$$ O_{decode}(p * q * (m + n)) $$

\begin{verbatim}
m := quantidade de caracteres da expressão regex
n := quantidade de caracteres da string do METAR a ser analisado
p := quantidade de itens do METAR
q := quantidade de expressões regex no programa
\end{verbatim}


A maior expressão regex no decoder é \texttt{([A-Z]{3})(\d{3})(CB|TCU)*} com 26 caracteres.
Para efeitos práticos este é um valor muito pequeno então podemos assumir m constante.

Sabemos que o número de expressões é 11, também um valor que pode ser assumindo contante, logo
q é igual à 1, portanto.

$$ O_{decode}(p * 1 * (1 + n)) $$
$$ O_{decode}(p * n) $$

Apenas as variáveis "p" e "n" dependem de valores externos. 
