\chapter{Rotas do back-end}

A seguir são listadas e explicadas todas as rotas do backend.

\section{Rota raiz}

Página inicial, apresenta a lista de aeródromos para que o usuário escolha um. 
Internamente, por meio do ORM, é feita a seleção dos campos \texttt{AerodromeName},
\texttt{ICAO} e \texttt{City} e o resultado é posto em uma lista de tuplas que é 
enviada para o template.

Exemplo do resultado do banco:
\begin{verbatim}
[
    ('Presidente Juscelino Kubitschek', 'SBBR', 'Brasília'), 
    ('Tancredo Neves', 'SBCF', 'Belo Horizonte'), 
    ('Afonso Pena', 'SBCT', 'Curitiba'), 
    ('Pinto Martins', 'SBFZ', 'Fortaleza'), 
    ...
]
\end{verbatim}

\section{Rota: /info/\{ICAO\}}

Retorna informações de um aeródromo com o ICAO especificado na URL. São exibidos,
além da explicação do METAR atual:

\begin{itemize}
    \item Pista
        \begin {itemize}
        \item Cabeceiras
        \item Comprimento
        \item Largura.
        \end {itemize}
    \item Frequências do aeródromo (nem todos os items a seguir poderão estar disponíveis). Visualmente cada item contém o tipo a que se refere e a frequência em si.
    \begin{itemize}
        \item Torre
        \item Solo
        \item Operações
        \item Rampa
        \item Tráfego
        \item ATIS
    \end{itemize}
    \item Frequências de navegação (nem todos os items a seguir poderão estar disponíveis)
    \begin{itemize}
        \item ILS
            \begin{itemize}
                \item Qual cabeceira este ILS se refere
                \item Frequência
                \item Direção final de aproximação (CRS)
                \item Identificador (um código de três letras que este ILS é identificado nas cartas aeronauticas)
            \end{itemize}
        \item VOR
            \begin{itemize}
                \item Frequência
                \item Identificador (um código de três letras que este VOR é identificado nas cartas aeronauticas)
            \end{itemize}
    \end{itemize}

\end{itemize}

\section{Rota: /history/\{ICAO\}}
Para um aeródromo retorna informações dos dez últimos METARs em três
gráficos com o eixo horizonal sendo o tempo.

\begin{itemize}
    \item Gráfico 1
    \begin{itemize}
        \item Temperature (graus Célsius)
        \item Ponto de orvalho (graus Célsius)
    \end{itemize}
    \item Gráfico 2
    \begin{itemize}
        \item Velocidade do vento (milhas náuticas por hora)
        \item Direção do vento (graus)
    \end{itemize}
    \item Gráfico 3
    \begin{itemize}
        \item Ajuste altímetro (hectopascal)
        \item Visibilidade (metro)
    \end{itemize}
\end{itemize}

\section{Rota: /taf/\{ICAO\}}
Retorna o próximo TAF válido para este aeródromo com a explicação de cada item.


